% LREC-COLING 2024 has both long and short papers featuring substantial, original, and unpublished research in all aspects of natural language and computation, language resources (LRs) and evaluation, including spoken and sign language and multimodal interaction. Submissions are invited in five broad categories: (i) theories, algorithms, and models, (ii) NLP applications, (iii) language resources, (iv) NLP evaluation and (v) topics of general interest. Submissions that span multiple categories are particularly welcome.

% \begin{itemize}
%     \item{The paper is in A4-size format, that is 21 x 29.7 cm.}
%     \item{The text height is 24.7 cm and the text width 16.0 cm in two columns separated by a 0.6 cm space.}
%      \item {The font for the main body of the text must be Arial 10 pt with interlinear spacing of 11 pt.}
%      \item {The use of LREC-COLING2024.sty will ensure the good formatting.}
% \end{itemize}


% \paragraph{Submissions may be of three types:}

% \begin{itemize}
% \item Regular long papers - up to eight (8) pages maximum,* presenting substantial, original, completed, and unpublished work.
% \item Short papers - up to four (4) pages,\footnote{Excluding any number of additional pages for references, ethical consideration, conflict-of-interest, as well as data and code availability statements.} describing a small, focused contribution, negative results, system demonstrations, etc.
% \item  Position papers - up to eight (8) pages,* discussing key hot topics, challenges and open issues, and cross-fertilization between computational linguistics and other disciplines.
% \end{itemize}

% Upon acceptance, final versions of long papers will be given one
% additional page – up to nine (9) pages of content plus unlimited pages for acknowledgments and references – so that reviewers’ comments can be considered. Final versions of short papers may have up to five (5) pages, plus unlimited pages for acknowledgments and references. All figures and tables that are part of the main text must fit within these page limits for long and short papers.

% Papers must be of original, previously-unpublished work. Papers must be \textbf{anonymized to support double-blind reviewing}. Submissions, thus, must not include authors’ names and affiliations. The submissions should also avoid links to non-anonymized repositories: the code should be either submitted as supplementary material in the final version of the paper or as a link to an anonymized repository (e.g., Anonymous GitHub or Anonym Share). Papers that do not conform to these requirements will be rejected without review.



% \section{How to Produce the \texttt{.pdf}}
% \label{sec:append-how-prod}


% In order to generate a PDF file out of the LaTeX file herein, when citing language resources, the following steps need to be performed:

% \begin{enumerate}
% \item \texttt{xelatex your\ paper\ name.tex}
% \item \texttt{bibtex your\ paper\ name.aux}
% \item \texttt{bibtex languageresource.aux}    *NEW*
% \item \texttt{xelatex your\ paper\ name.tex}
% \item \texttt{xelatex your\ paper\ name.tex}
% \end{enumerate}

% From 2024 we are using the ARIAL font, so you must install it if you
% do not have it (for example from the mscorefonts package).  To compile, you must use either \textbf{\texttt{XeTeX}} or \textbf{\texttt{LuaTeX}}.

% \section{ Final Paper}

% Each final paper should be submitted online. The fully justified text should be formatted according to LREC-COLING2024 style as indicated for the Full Paper submission.

% As indicated above, the font for the main body of the text should be ARIAL 10 pt with interlinear spacing of 11 pt. Papers must be between 4 and 8 pages long, including figures (plus more pages for references if needed), regardless of the presentation mode (oral or poster).

% \subsection{General Instructions for the Final Paper}

% The unprotected PDF files will appear in the online proceedings directly as received. \textbf{Do not print the page number}.

% \section{Page Numbering}

% \textbf{Please do not include page numbers in your Paper.} The definitive page numbering of papers published in the proceedings will be decided by the Editorial Committee.

% \section{Headings / Level 1 Headings}

% Level 1 Headings should be capitalised in the same way as the main title, and centered within the column. The font used is Arial 12 pt bold. There should also be a space of 12 pt between the title and the preceding section and 3 pt between the title and the following text.

% \subsection{Level 2 Headings}

% The format of Level 2 Headings is the same as for Level 1 Headings, with the font Arial 11 pt, and the heading is justified to the left of the column. There should also be a space of 6 pt between the title and the preceding section and 3 pt between the title and the following text.

% \subsubsection{Level 3 Headings}
% \label{level3H}

% The format of Level 3 Headings is the same as Level 2, except that the font is Arial 10 pt, and there should be no space left between the heading and the text as in \ref{level3H}. There should also be a space of 6 pt between the title and the preceding section and 3 pt between the title and the following text.

% \section{Citing References in the Text}

% \subsection{Bibliographical References}


% \begin{table}
% \centering
% \begin{tabular}{lll}
% \hline
% \textbf{Output} & \textbf{natbib command} & \textbf{Old command}\\
% \hline
% \citep{Eco:1990} & \verb|\citep| & \verb|\cite| \\
% \citealp{Eco:1990} & \verb|\citealp| & no equivalent \\
% \citet{Eco:1990} & \verb|\citet| & \verb|\newcite| \\
% \citeyearpar{Eco:1990} & \verb|\citeyearpar| & \verb|\shortcite| \\
% \hline
% \end{tabular}
% \caption{\label{citation-guide} Citation commands supported by the style file. The style is based on the natbib package and supports all natbib citation commands. It also supports commands defined in previous style files for compatibility.}
% \end{table}

% Table~\ref{citation-guide} shows the syntax supported by the style files. We encourage you to use the natbib styles.
% You can use the command \verb|\citet| (cite in text) to get ``author (year)'' citations, like this citation to a paper by \citet{CastorPollux-92}. You can use the command \verb|\citep| (cite in parentheses) to get ``(author, year)'' citations \citep{CastorPollux-92}. You can use the command \verb|\citealp| (alternative cite without parentheses) to get ``author, year'' citations, which is useful for using citations within parentheses (e.g. \citealp{CastorPollux-92}).

% When several authors are cited, those references should be separated with a semicolon: \cite{Martin-90,CastorPollux-92}. When the reference has more than three authors, only cite the name of the first author followed by ``et. al.'', e.g. \cite{Superman-Batman-Catwoman-Spiderman-00}.

% \subsection{Language Resource References}

% \subsubsection{When Citing Language Resources}

% As you may know, LREC introduced a separate section on Language Resources citation to enhance the value of such assets. When citing language resources, we recommend to proceed in the same way as for bibliographical references. Please make sure to compile your Language Resources stored as a .bib file \textbf{separately} (BibTex under XeTeX). This produces the required .aux et .bbl files. Thus, a language resource should be cited as \citetlanguageresource{Speecon} or \citeplanguageresource{EMILLE} .

% See Section~\ref{sec:append-how-prod} for details on how to produce this with bibtex.

% \section{Figures \& Tables}

% \subsection{Figures}

% All figures should be centred and clearly distinguishable. They should never be drawn by hand, and the lines must be very dark in order to ensure a high-quality printed version. Figures should be numbered in the text, and have a caption in Arial 10 pt underneath. A space must be left between each figure and its respective caption.

% Example of a figure:

% \begin{figure}[!ht]
% \begin{center}
% \includegraphics[scale=0.5]{turin2024-banner.jpg}
% \caption{The caption of the figure.}
% \label{fig.1}
% \end{center}
% \end{figure}

% Figure and caption should always appear together on the same page. Large figures can be centered, using a full page.

% \subsection{Tables}

% The instructions for tables are the same as for figures.

% \begin{table}[!ht]
% \begin{center}
% \begin{tabularx}{\columnwidth}{|l|X|}

%       \hline
%       Level&Tools\\
%       \hline
%       Morphology & Pitrat Analyser\\
%       \hline
%       Syntax & LFG Analyser (C-Structure)\\
%       \hline
%      Semantics & LFG F-Structures + Sowa's\\
%      & Conceptual Graphs\\
%       \hline

% \end{tabularx}
% \caption{The caption of the table}
%  \end{center}
% \end{table}

% \section{Footnotes}

% Footnotes are indicated within the text by a number in superscript\footnote{Footnotes should be in Arial 9 pt, and appear at the bottom of the same page as their corresponding number. Footnotes should also be separated from the rest of the text by a 5 cm long horizontal line.}.

% \section{Copyrights}

% The Language Resouces and Evaluation Conference (LREC) Proceedings are published by the European Language Resources Association (ELRA). They are available online from the conference website.

% ELRA's policy is to acquire copyright for all LREC contributions. In assigning your copyright, you are not forfeiting your right to use your contribution elsewhere. This you may do without seeking permission and is subject only to normal acknowledgment to the LREC proceedings. The LREC Proceedings are licensed under CC-BY-NC, the Creative Commons Attribution-Non-Commercial 4.0 International License.


% \section{Conclusion}

% Your submission of a finalized contribution for inclusion in the LREC Proceedings automatically assigns the above copyright to ELRA.

% \section{Acknowledgements}

% Place all acknowledgments (including those concerning research grants and funding) in a separate section at the end of the paper.

% \section{Optional Supplementary Materials}

% Appendices or supplementary material (software and data) will be allowed ONLY in the final, camera-ready version, but not during submission, as papers should be reviewed without the need to refer to any supplementary
% materials.

% Each \textbf{camera ready} submission can be accompanied by an appendix usually being included in a main PDF paper file, one \texttt{.tgz} or \texttt{.zip} archive containing software, and one \texttt{.tgz} or \texttt{.zip} archive containing data.

% We encourage the submission of these supplementary materials to improve the reproducibility of results and to enable authors to provide additional information that does not fit in the paper. For example, preprocessing decisions, model parameters, feature templates, lengthy proofs or derivations, pseudocode, sample system inputs/outputs, and other details necessary for the exact replication of the work described in the paper can be put into the appendix. However, the paper submissions must remain fully self-contained, as these supplementary materials are optional, and reviewers are not even asked to review or download them. If the pseudo-code or derivations, or model specifications are an essential part of the contribution, or if they are important for the reviewers to assess the technical correctness of the work, they should be a part of the main paper and not appear in the appendix. Supplementary materials need to be fully anonymized to preserve the double-blind reviewing policy.

% \subsection{Appendices}

% Appendices are material that can be read and include lemmas, formulas, proofs, and tables that are not critical to the reading and understanding of the paper, as in \href{https://acl-org.github.io/ACLPUB/formatting.html#appendices}{*ACLPUB}. It is  highly recommended that the appendices should come after the references; the main text and appendices should be contained in a `single' manuscript file, without being separately maintained. Letter them in sequence and provide an informative title: \textit{Appendix A. Title of Appendix}


% \subsection{Extra space for ethical considerations and limitations}

% Please note that extra space is allowed after the 8th page (4th page for short papers) for an ethics/broader impact statement and a discussion of limitations. At submission time, if you need extra space for these sections, it should be placed after the conclusion so that it is possible to rapidly check that the rest of the paper still fits in 8 pages (4 pages for short papers). Ethical considerations sections, limitations, acknowledgments, and references do not count against these limits. For camera-ready versions, nine pages of content will be allowed for long (5 for short)
% papers.

% \section{Providing References}

% \subsection{Bibliographical References}

% Bibliographical references should be listed in alphabetical order at the end of the paper. The title of the section, ``Bibliographical References'', should be a Level 1 Heading. The first line of each bibliographical reference should be justified to the left of the column, and the rest of the entry should be indented by 0.35 cm.

% The examples provided in Section~\ref{sec:reference} (some of which are fictitious references) illustrate the basic format required for papers in conference proceedings, books, journal articles, PhD theses, and books chapters.

% \subsection{Language Resource References}

% Language resource references should be listed in alphabetical order at the end of the paper.



%%% Local Variables:
%%% mode: latex
%%% TeX-master: t
%%% End: